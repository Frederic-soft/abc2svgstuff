\documentclass[a4paper,twoside]{article}
\usepackage[T1]{fontenc}
\usepackage[utf8]{inputenc}
\usepackage[%
  a4paper,
  top=1.5cm, bottom=1.5cm,
  left=1.5cm,right=2.5cm,
  bindingoffset=0.5cm
]{geometry}
\usepackage{calc}
\usepackage{pict2e}
% \usepackage{fourier}
\usepackage{times}
\usepackage{helvet}
\usepackage{courier}
\usepackage{xspace}
\usepackage[french]{babel}
\usepackage[%
%  keepfiles,   % keep ABC and PDF files between runs
  reusefiles,  % reuse ABC and PDF files from previous runs
%                %  (implies keepfiles)
]{abcsvgintex}
\abcsvgauxdir{./abcsvgtmp/}

% A prefix (to be used in \prelude=\choral) for chorals.
% Four voices are defined: Soprano, Alto, Tenor and Bass.
% Soprano and Alto use the treble clef on the top staff, Tenor and 
% Bass use the bass clef on the lower staff. The 'I' fields tell 
% abc2midi that notes in the Tenor and Bass voices are to be played 
% two octaves lower (for abcm2ps, we write "a" instead of "A," for 
% the A which is on the upper line of the staff with bass clef).
\begin{abcsvgprefix}{\choral}
  V:S clef=treble stem=up dyn=up
  V:A clef=treble stem=down dyn=up
  V:T clef=bass stem=up dyn=down octave=-2
  I:  clef=bass octave=-2
  V:B clef=bass stem=down dyn=down octave=-2
  I:  clef=bass octave=-2
  %%staves {(S A) (T B)}
\end{abcsvgprefix}

% A prefix (to be used in \prelude=\piano) for piano.
% Two voices are defined, one (D) for the right hand, using the 
% treble clef, and the other (G) for the left hand, using the bass 
% clef.
\begin{abcsvgprefix}{\piano}
  V:D clef=treble stem=up dyn=up
  V:G clef=bass stem=down dyn=down
  I:  clef=bass octave=-2
  %%staves {D G}
\end{abcsvgprefix}

% A LaTeX savebox to put music in.
\newsavebox{\musicbox}

\setlength{\parindent}{0pt}
\newlength{\staffwidth}
\newcommand{\accord}[3]{%
  \textrm{#1}\({\,}_{#2}^{#3}\)%
}
\newcommand{\tbox}[1]{%
  \makebox(0,0)[t]{\sffamily#1}
}
\newcommand{\bbox}[1]{%
  \makebox(0,0)[b]{\sffamily#1}
}
\newcommand{\pbox}[1]{%
  \hbox to 0pt{\hss\parbox[b]{\linewidth}{\sffamily\centering#1}\hss}%
}
\newcommand{\rcrochet}[1]{%
  \begin{picture}(0,0)(0,0)
    \put(0,0){\line(1,0){5}}
    \put(5,0){\line(0,-1){#1}}
    \put(5,-#1){\line(-1,0){5}}
  \end{picture}
}
\newcommand{\lcrochet}[1]{%
  \begin{picture}(0,0)(0,0)
    \put(0,0){\line(-1,0){5}}
    \put(-5,0){\line(0,-1){#1}}
    \put(-5,-#1){\line(1,0){5}}
  \end{picture}
}

\begin{document}

\begin{center}
  \LARGE\bfseries\itshape 7\ieme de dominante en majeur
\end{center}

\bigskip

Dans l'accord de 7\ieme de dominante, on peut éliminer la quinte et
doubler la fondamentale quel que soit le renversement. Dans la
notation \accord{}+7, le signe \(+\) indique la sensible qui doit
obligatoirement trouver sa résolution dans l'accord suivant comme
indiqué ci-après.

\smallskip

% We want staves almost as wide as a line
\setlength{\staffwidth}{0.99\linewidth}
\abcsvgpdfsettings{%
%  staffwidth=\the\staffwidth,  % set the width of the staff
  pagewidth=\the\staffwidth,  % set the width of the staff
  leftmargin=0pt,
  rightmargin=0pt,
  linebreak=<none>,            % don't obey lines in the ABC source
  stretchlast=1
}
\abcsvgsettings{
  prelude=\piano,   % our setup for piano
  M=none,           % no meter
  L=1/1,            % unit length is a whole note
  K=C,              % Key is C Major
}
% Save the output of abcm2ps into \musicbox
\begin{abcsvgbox}{\musicbox}
  [V:D] y[GBdf] | y[B,DFG] | y[DFGB] | y[FGBd] ||
  [V:G]  x      |  x       |  x      |  x      ||
  %
  [V:D] y[DF]   | y[DF]    | y[B,F]  | y[B,D]  |]
  [V:G]  [G,B,] |  [B,,G,] |  [D,G,] |  [F,G,] |]
\end{abcsvgbox}
% annotate this output
\begin{abcsvgannotate}[%
   bottom=0.1\height+\baselineskip% leave room at the bottom for 
                                  % annotations under the staff
  ]{\musicbox}
  \rput(0.1,-0.1){\tbox{État fond.}}
  \rput(0.225,-0.1){\tbox{1\ier renv.}}
  \rput(0.35,-0.1){\tbox{2\ieme renv.}}
  \rput(0.48,-0.1){\tbox{3\ieme renv.}}
  \rput(0.585,-0.1){\tbox{\accord V+7}}
  \rput(0.705,-0.1){\tbox{\accord V{\not5}6}}
  \rput(0.83,-0.1){\tbox{\accord V{}{+6}}}
  \rput(0.94,-0.1){\tbox{\accord V{}{+4}}}
\end{abcsvgannotate}

\bigskip

\abcsvgsettings{
  prelude=\choral,
  M=none,
  L=1/4,
  K=C
}
\begin{abcsvgbox}{\musicbox}
  [V:S] y (F  |  F4) E4  || F4  G4 || B4  c4  || B4  c4  |]
  [V:A] y  C  |  D4  C4  || D4  C4 || F4  E4  || F4  G4  |]
  [V:T] y  a  | (g4  g4) || b4  g4 || d'4 c'4 || d'4 c'4 |]
  [V:B] y  f  |  B4  c4  || g4  e4 || g4  c4  || g4  e4  |]
  %
\end{abcsvgbox}

\begin{abcsvgannotate}[top=0.2\height,bottom=0.1\height+\baselineskip]{\musicbox}
  \rput(0.1,1){\pbox{\small préparation}}
  %
  \rput(0.23,1){\pbox{\small résolution\\normale}}
  \thicklines
  \rput(0.165,0.7){\line(15,-1){35}}
  %
  \rput(0.445,1){\pbox{\small résolution\\exceptionnelle}}
  \rput(0.39,0.69){\line(15,1){34}}
  \rput(0.39,0.64){\line(9,-8.5){34}}
  %
  \rput(0.615,0.78){\line(15,1){34}}
  \rput(0.615,0.68){\line(15,-1){34}}
  %
  \rput(0.84,0.69){\line(15,1){34}}
  \rput(0.84,0.66){\line(6,-6){34}}
  %
  \rput(0.08,-0.1){\tbox{\accord{IV}{}{}}}
  \rput(0.155,-0.1){\tbox{\accord V{\not5}6}}
  \rput(0.26,-0.1){\tbox{\accord I{}5}}
  \rput(0.385,-0.1){\tbox{\accord V+7}}
  \rput(0.49,-0.1){\tbox{\accord I{}6}}
  \rput(0.61,-0.1){\tbox{\accord V+7}}
  \rput(0.714,-0.1){\tbox{\accord I{}5}}
  \rput(0.835,-0.1){\tbox{\accord V+7}}
  \rput(0.94,-0.1){\tbox{\accord I{}6}}
\end{abcsvgannotate}

\medskip

\begin{itemize}
  \item Résolution normale~: la 7\ieme descend en mouvement conjoint~;
  \item Résolution exceptionnelle~: la 7\ieme monte en mouvement
  conjoint, et la basse fait entendre sa résolution normale.
\end{itemize}

\abcsvgsettings{
  prelude=\choral,
  M=none,
  L=1/4,
  K=C,
}
\abcsvgpdfsettings{
  stretchlast=0
}

\begin{abcsvgbox}{\musicbox}
  [V:S] y  G y F |  F y G  | C y F  ||
  [V:A] y  D y D |  D y D  | C y B, ||
  [V:T] y  b y b |  b y b  | a y g  ||
  [V:B] y  f y g |  g y f  | f y g  ||
  [V:S][L:1/1] A  F | G  F | E  F ||
  [V:A][L:1/1] F  D | E  D | C  D ||
  [V:T][L:1/1] c' b | c' b | g  b ||
  [V:B][L:1/1] f  g | c  G | c  g ||
  [V:S][L:1/4] E  D  G2 |]
  [V:A][L:1/4] C  B, C2 |]
  [V:T][L:1/4] g  g  g2 |]
  [V:B][L:1/4] c  f  e2 |]
\end{abcsvgbox}

\medskip

\begin{abcsvgannotate}[%
%   grid, % the grid can help placing annotations
  top=2\baselineskip,
  bottom=0.1\height+\baselineskip
  ]{\musicbox}
  \rput(0.21,1.04){\pbox{%
    \small mouvements déconseillés entre\\
    la 7\ieme au soprano et la fondamentale à la basse
  }}
  \thicklines
  \rput(0.085,0.73){\line(15,-1){18}}
  \rput(0.085,0.3){\line(15,1){18}}
  \rput(0.145,0.71){\rcrochet{30}}
  %
  \rput(0.08,-0.04){\tbox{\accord V{}{+4}}}
  \rput(0.14,-0.04){\tbox{\accord V+7}}
  %
  \rput(0.195,0.72){\line(15,1){18}}
  \rput(0.195,0.32){\line(15,-1){18}}
  \rput(0.26,0.74){\rcrochet{35}}
  %
  \rput(0.19,-0.04){\tbox{\accord V+7}}
  \rput(0.25,-0.04){\tbox{\accord V{}{+4}}}
  %
  \rput(0.31,0.635){\line(3,1){16}}
  \rput(0.31,0.27){\line(15,1){16}}
  %
  \rput(0.30,-0.04){\tbox{\accord{IV}{}5}}
  \rput(0.36,-0.04){\tbox{\accord V+7}}
  %
  %%%
  \rput(0.63,1.04){\pbox{%
    \small mouvement contraire ou direct possible
  }}
  \rput(0.45,0.78){\line(8,-1){18}}
  \rput(0.45,0.24){\line(14,1){18}}
  %
  \rput(0.44,-0.04){\tbox{\accord{IV}{}5}}
  \rput(0.515,-0.04){\tbox{\accord V+7}}
  %
  \rput(0.59,-0.04){\tbox{\accord I{}5}}
  \rput(0.66,-0.04){\tbox{\accord V+7}}
  %
  \rput(0.735,-0.04){\tbox{\accord I{}5}}
  \rput(0.81,-0.04){\tbox{\accord V+7}}
  %%%
  \rput(0.92,1.04){\pbox{%
    \small sans préparation\\(chez Beethoven)
  }}
  %
  \rput(0.875,-0.04){\tbox{\accord I{}5}}
  \rput(0.92,-0.04){\tbox{\accord V{}{+4}}}
  \rput(0.96,-0.04){\tbox{\accord I{}6}}
\end{abcsvgannotate}

On peut enchaîner quinte juste et quinte diminuée. L'inverse n'est pas
possible, sauf si la basse effectue un mouvement de sixte et de tierce
parallèle (faux-bourdon) avec deux autres voix~:

\smallskip
\abcsvgsettings{
  prelude=\choral,
  M=none,
  L=1/1,
  K=C,
}
\begin{abcsvgbox}{\musicbox}
  [V:S] yy  F  G |]
  [V:A] yy  B, C |]
  [V:T] yy  g  g |]
  [V:B] yy  d  e |]
\end{abcsvgbox}
\begin{abcsvgannotate}[%
%     grid,
    bottom=0.1\height+\baselineskip
  ]{\musicbox}
  \thicklines\sffamily
  \rput(0.46,0.68){\lcrochet{12}}
  \rput(0.25,0.55){5\(-\)}
  \rput(0.85,0.712){\rcrochet{12.5}}
  \rput(1.05,0.57){5J}
  \rput(0.58,0.64){\line(1,-1){25}}
  %
  \rput(0.55,-0.08){\tbox{\accord V{}{+6}}}
  \rput(0.80,-0.08){\tbox{\accord I{}6}}
\end{abcsvgannotate}

Résolution figurée~:

\smallskip
\abcsvgsettings{
  prelude=\choral,
  M=none,
  L=1/8,
  K=C,
}
\begin{abcsvgbox}{\musicbox}
  [V:S] y  FG | E4 || B2 | c4 |]
  [V:A] y  D2 | C4 || FD | E4 |]
  [V:T] y  b2 | g4 || g2 | g4 |]
  [V:B] y  g2 | c4 || G2 | c4 |]
\end{abcsvgbox}
\begin{abcsvgannotate}[bottom=0.05\height+\baselineskip]{\musicbox}
  \rput(0.195,-0.05){\tbox{\accord V+7}}
  \rput(0.285,-0.05){\tbox{(\accord V{}5)}}
  \rput(0.420,-0.05){\tbox{\accord I{}5}}
  \rput(0.622,-0.05){\tbox{\accord V+7}}
  \rput(0.72,-0.05){\tbox{(\accord V{}5)}}
  \rput(0.842,-0.05){\tbox{\accord I{}5}}
\end{abcsvgannotate}

\section*{Exercices}
% Here, we define a new decoration !*!, that draws a star below a note
% We first define the PostScript procedure that draws the star, then
% we declare a new decoration that uses this procedure.
\newtoks\choralwithstar % define a new token register to be used as a prefix
\choralwithstar=\choral % set it to be the same as our \choral prefix
\begin{abcsvgmoreprefix}{\choralwithstar} % then, add stuff to it
  %%beginps
  /domstar{M -5 -74 RM /Helvetica 24 selectfont(*)show}!
  %%endps
  %%deco * 6 domstar 0 0 0 
\end{abcsvgmoreprefix}

\raisebox{-.5\height}{\textsf{\LARGE*}}~:
utilisez éventuellement l'accord de 7\ieme de dominante.
\abcsvgpdfsettings{
  scale=0.65,   % smaller scale so that all exercises fit on the page
  linebreak=<EOL>,% obey lines in ABC source
  stretchlast=1
}%
\nopagebreak

\medskip
\fbox{Ré majeur}\par\nopagebreak
\abcsvgsettings{
  prelude=\choralwithstar,
  M=4/4,
  L=1/4,
  K=DMaj,
}
\begin{abcsvg}
  [V:S] dcde | fgfe | dAGF | E2AG | FGAd |
  s:         | *!*! | **!*!| **!*!|      |
  [V:A] x4   | x4   | x4   | x4   | x4   |
  [V:T] x4   | x4   | x4   | x4   | x4   |
  [V:B] x4   | x4   | x4   | x4   | x4   |
  [V:S] cd2g-| gfcd | e2AG | FAdc | d4   |]
  s:    !*!  | !*!  | **!*!|*!*!*!*!|    |
  [V:A] x4   | x4   | x4   | x4   | x4   |]
  [V:T] x4   | x4   | x4   | x4   | x4   |]
  [V:B] x4   | x4   | x4   | x4   | x4   |]
\end{abcsvg}

\begin{center}
  \rule{0.3\linewidth}{1pt}
\end{center}

\fbox{Fa majeur}\par\nopagebreak
\abcsvgsettings{
  prelude=\choral,
  M=4/4,
  L=1/4,
  K=FMaj,
}
\begin{abcsvgbox}{\musicbox}
  [V:S] x4      | x4   | x4     | x4   | x4   | x4   | x4     | x4   |]
  [V:A] x4      | x4   | x4     | x4   | x4   | x4   | x4     | x4   |]
  [V:T] x4      | x4   | x4     | x4   | x4   | x4   | x4     | x4   |]
  [V:B] F2 cB   | Acfe | fBc2   | fgac | defA | BcdG | ABc2   | F4   |]
\end{abcsvgbox}
\begin{abcsvgannotate}[bottom=0.05\height+\baselineskip]{\musicbox}
  \rput(0.175,-0.05){\tbox{\accord V{}{+4}}}
  \rput(0.410,-0.05){\tbox{\accord V+7}}
  \rput(0.612,-0.05){\tbox{\accord V{\not5}6}}
  \rput(0.795,-0.05){\tbox{\accord V{}{+6}}}
  \rput(0.900,-0.05){\tbox{\accord V+7}}
\end{abcsvgannotate}

\begin{center}
  \rule{0.3\linewidth}{1pt}
\end{center}

\fbox{La majeur}\par\nopagebreak
\abcsvgsettings{
  prelude=\choralwithstar,
}
% In this example, we set the meter, note length and key in the ABC 
% source, not with \abcsettings, just to show that it is still 
% possible
\begin{abcsvg}
  M:4/4
  L:1/4
  K:Amaj
  [V:S] AGAE | cBAG | Acdc | BAG2 | Afed |
  s:         |***!*!|      |**!*! |***!*!|
  [V:A] x4   | x4   | x4   | x4   | x4   |
  [V:T] x4   | x4   | x4   | x4   | x4   |
  [V:B] x4   | x4   | x4   | x4   | x4   |
  %
  [V:S] cdeA | dcB2 | AEFE-| EDCE | AdcB | A4 |]
  s:         | **!*!|      | !*!  |***!*!|    |
  [V:A] x4   | x4   | x4   | x4   | x4   | x4 |]
  [V:T] x4   | x4   | x4   | x4   | x4   | x4 |]
  [V:B] x4   | x4   | x4   | x4   | x4   | x4 |]
\end{abcsvg}

\begin{center}
  \rule{0.3\linewidth}{1pt}
\end{center}

\choralwithstar=\choral
\begin{abcsvgmoreprefix}{\choralwithstar}
  %%beginps
  /domstar{M -5 47 RM /Helvetica 24 selectfont(*)show}!
  %%endps
  %%deco * 6 domstar 0 0 0 
\end{abcsvgmoreprefix}
\fbox{Mi majeur}\par\nopagebreak
\abcsvgsettings{
  prelude=\choralwithstar,
}
\begin{abcsvg}
  M:4/4
  L:1/4
  K:Emaj
  [V:S] x4   | x4   | x4   | x4   | x4   | x4   |
  [V:A] x4   | x4   | x4   | x4   | x4   | x4   |
  [V:T] x4   | x4   | x4   | x4   | x4   | x4   |
  [V:B] e2BA | GBef | gaba | g2fd | eABG | AGFE |
  s:    **!*!|      |***!*!|      |**!*! |      |
  %
  [V:S] x4   | x4   | x4   | x4   | x4   | x4   |]
  [V:A] x4   | x4   | x4   | x4   | x4   | x4   |]
  [V:T] x4   | x4   | x4   | x4   | x4   | x4   |]
  [V:B] BAGB | edec | B2ba | gbea | g2fB | e4   |]
  s:    *!*! | *!*! | **!*!|*!*!*!*!|*!*!!*!|   |
\end{abcsvg}

\end{document}
