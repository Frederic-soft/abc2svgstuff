\documentclass[a4paper]{article}
\usepackage[T1]{fontenc}
\usepackage[utf8]{inputenc}
\usepackage[top=2cm,left=2cm]{geometry}
\usepackage{calc}
\usepackage[%
  keepfiles
]{abcsvgintex}
\abcsvgauxdir{./abcsvgtmp/}
\newsavebox{\mybox}

\begin{abcsvgprefix}{\choral}
  V:S clef=treble stem=up dyn=up
  V:A clef=treble stem=down dyn=up
  V:T clef=bass stem=up dyn=down octave=-2
  I:  clef=bass octave=-2
  V:B clef=bass stem=down dyn=down octave=-2
  I:  clef=bass octave=-2
  %%staves {(S A) (T B)}
\end{abcsvgprefix}

\begin{document}
Une page de musique :

\abcsvgsettings{%
  raise=-0.5\height+0.5ex,
  prelude=\choral
}%
\abcsvgpdfsettings{pagewidth=10cm,maxshrink=0.5,measurenb=-1,printtempo=0}%
\noindent La musique :
\begin{abcsvg}
  M:C
  L:1/4
  Q:1/4=60
  K:C
  [V:S] c d e f  | d B c2        |]
  [V:A] G3    A  | G F E/2F/2  E |]
  [V:T] e'd'c'd'-| d'2 c'/2a/2 g |]
  [V:B] c B c A  | B G c2        |]
\end{abcsvg}
\ adoucit les mœurs.

\noindent Une boîte 
\abcsvgsettings{%
  prelude=\relax
}%
\begin{abcsvgbox}{\mybox}
  T: Un essai
  C: Frédéric Boulanger
  M:C
  L:1/4
  K:C
  [CE][DF][EG][FA] |]
\end{abcsvgbox}
\raisebox{-\height}{\fboxsep0pt\fbox{\usebox{\mybox}}}
à musique.

Voilà !

\end{document}
